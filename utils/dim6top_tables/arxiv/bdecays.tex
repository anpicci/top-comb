% mainfile: eft_note.tex

{
\let\section\subsection

%\documentclass[12pt,a4paper]{article}
%\pdfoutput=1


%\usepackage{graphicx,array}
%\usepackage{hyperref}
%\usepackage{comment}
%\usepackage[normalem]{ulem} 
%\usepackage{cite}
%\usepackage{color}
%\usepackage{appendix}
%\usepackage{epsfig}
%\usepackage{latexsym}
%\usepackage{amsmath}
%\usepackage{amssymb}
%\usepackage{bm}
%\usepackage{hyperref}
%\usepackage{slashed}
%\usepackage{bbold}
%\usepackage{subfig}
%\usepackage{multirow}

%\topmargin=-0.5cm
%\oddsidemargin=0.0 in
%\evensidemargin=0.0in
%\textheight=21cm
%\textwidth=16cm

%\definecolor{darkblue}{cmyk}{1,0.3,0,0.2}
%\definecolor{violet}{cmyk}{0,1,0,0.2}
%\hypersetup{colorlinks, bookmarksnumbered, citecolor=darkblue, linkcolor=darkblue, pdfstartview=FitH, urlcolor=darkblue, linktocpage}

%\newcommand{\david}[1]{{\color{red} #1}}

%% For the link to arXiv in the references
%\newcommand{\arXhref}[1]{\href{http://arxiv.org/abs/#1}{#1}}

\newcommand{\be}{\begin{equation}}
\newcommand{\ee}{\end{equation}}
\newcommand{\bea}{\begin{eqnarray}}
\newcommand{\eea}{\end{eqnarray}}
%\newcommand{\Eq}[1]{Eq.~(\ref{#1})}
%\newcommand{\Eqs}[2]{Eqs.~(\ref{#1}) and (\ref{#2})}
%\newcommand{\Sec}[1]{Sect.~\ref{#1}}
%\newcommand{\Secs}[2]{Secs.~\ref{#1} and \ref{#2}}
%\newcommand{\Fig}[1]{Fig.~(\ref{#1})}
%\newcommand{\Figs}[2]{Figs.~\ref{#1} and \ref{#2}}
%\newcommand{\App}[1]{App.~\ref{#1}}
%\newcommand{\vev}[1]{\langle #1 \rangle}
%\newcommand{\MeV}{\textrm{ MeV}}
\newcommand{\GeV}{\textrm{ GeV}}
\newcommand{\TeV}{\textrm{ TeV}}
\newcommand{\SU}{\textrm{SU}}
%\newcommand{\SO}{\textrm{SO}}
%\newcommand{\U}{\textrm{U}}
%\newcommand{\Tr}{\textrm{Tr}}
%\newcommand{\SM}{\textrm{SM}}
%\newcommand{\gsim}{\lower.7ex\hbox{$\;\stackrel{\textstyle>}{\sim}\;$}}
%\newcommand{\lsim}{\lower.7ex\hbox{$\;\stackrel{\textstyle<}{\sim}\;$}}
%\newcommand{\bs}[1]{\boldsymbol #1}
%\newcommand{\LL}{\mathcal{L}}
%\newcommand{\cL}{\mathcal{L}}
%\newcommand{\cM}{\mathcal{M}}
%\newcommand{\cU}{\mathcal{U}}
%\newcommand{\OO}{\mathcal{O}}
%\newcommand{\GG}{\mathcal{G}}
%\newcommand{\HH}{\mathcal{H}}
%\newcommand{\MM}{\mathcal{M}}

%\newcommand{\bN}{\bf{N}}
%\newcommand{\bNb}{\bf{\bar{N}}}
%\newcommand{\bS}{\bf{1}}
%\newcommand{\bD}{\bf{2}}
%\newcommand{\bT}{\bf{3}}
%\newcommand{\bTb}{\bf{\bar{3}}}
%\newcommand{\bE}{\bf{8}}

%\newcommand{\BR}{\mathcal{B}}
%%\newcommand{\BR}{\textrm{BR}}
%\newcommand{\draftnote}[1]{\textbf{#1}}
%\newcommand{\mPl}{m_{\rm Pl}}
%\newcommand{\sslash}[1]{\ensuremath\raisebox{-0.00cm}{{\small\slash}}\hspace{-0.21cm}#1\/}
%\newcommand{\dd}[1]{\frac{\partial}{\partial #1}}
%\newcommand{\sgn}{{\rm sgn}}

%\newcommand{\ba} {\begin{eqnarray}}
%\newcommand{\ea} {\end{eqnarray}}
%\newcommand{\vareps}{\varepsilon}
%\newcommand{\cpeps}{\epsilon^{\rm CP}}
%\newcommand{\no} {\nonumber}
%\newcommand{\cA} {\mathcal A}
%\newcommand{\cB} {\mathcal B}
%\newcommand{\cd} {{\cdot }}
%\newcommand{\GF} {G^{(\mu)}_F}
%\newcommand{\aem} {\alpha_{\rm em}}
%\newcommand{\mZ} {m_Z}
%\newcommand{\Dlr}{\stackrel{\leftrightarrow}{D}}

%\def\tr{\mathop{\rm tr}}
%\def\diag{\mathop{\rm diag}}
%\newcommand{\abs}[1]{\left|#1\right|}
%\renewcommand{\Re}{\mathop{\rm Re}}
%\renewcommand{\Im}{\mathop{\rm Im}}

\def\BK{\B(K^+\to \pi^+\nu\bar\nu)}
\def\BB{\B(B\to K^{(*)}\nu\bar\nu)}
\def\RD{R_{D^{(*)}}}




%-----------------------------------------------------------------------------------
%-----------------------------------------------------------------------------------
%\begin{document}
 
\section{Constraints from low-energy flavour physics}

The CKM matrix can be approximated as the identity matrix when focusing on measurements involving resonant top quarks. However, when confronting new physics models with flavour observables, suppressed in the SM by small CKM matrix elements, it is of course crucial to consistently keep all the CKM factors.
By working in the down-quark mass basis, the misalignment between the up- and down- quark masses can be described by considering the quark doublet as $q_i = (V^*_{ji} u_{L, j},  d_{L, i})^T$.
Another basis often used is the one where up quarks have diagonal mass matrix. This can be obtained by simply rotating $q_i$ with the CKM matrix: $\tilde q_i = (V q)_i = (u_{L, i},  V_{ij} d_{L, j})^T$.
This choice changes the flavour structure of the operators by the same CKM rotations, for example the term
$C^{(ij)} \bar q_i \gamma_\mu  q_j \to \tilde C^{(ij)} \bar {\tilde q}_i \gamma_\mu  {\tilde q}_j$, where $\tilde C = V C V^\dagger$.
%
In the spirit of this note, we assume that the $(33)$ element of such flavour matrices is the largest one. The $U(2)_q$ flavour symmetry relates the off-diagonal components to the CKM matrix \cite{Barbieri:2011ci,Barbieri:2012uh}. In this well motivated framework, in general the new physics contributions are not necessarily aligned with the up or down quarks, but will have some misalignment of the order of the relevant CKM elements.
%
For definiteness, in the following we mostly work in the down-quark mass basis.

Let us consider the charged-current transition $b \to c\, e_i \bar\nu_j$ and study the experimental limits we can extract on the top-quark operators. From the general effective Hamiltonian at the $B$-meson mass scale, we can consider only the operators involving the left-handed $c_L$ quark, since $c_R$ necessarily arises from a second-generation family index. We are left with
\be
	H_{\rm eff}^{b\to c e \bar\nu} = -\frac{2}{v^2} V_{cb} \left( (\delta^{i j} + c_{V_L}^{ij}) (\bar c_L \gamma^\mu b_L) (\bar e^i_L \gamma_\mu \nu_L^j ) + c_{S_R}^{ij} (\bar c_L b_R)(\bar e_R^i \nu_L^j) + h.c. \right)~,
\ee
where $v \approx 246 \GeV$. The tree-level matching to the SMEFT is given by \cite{Cirigliano:2012ab, Aebischer:2015fzz}
\be\begin{split}
	c_{V_L}^{ij} &= - \frac{v^2}{\Lambda^2} \left( \frac{\sum_{k} V_{ck} C_{l q}^{3 (ij k 3)}}{V_{cb}} \right) + \frac{v^2}{\Lambda^2} \left( \frac{\sum_{k} V_{ck} C_{\varphi q}^{3(k 3)} }{V_{cb}} \right)   \delta^{ij}~, \\
	c_{S_R}^{ij} &= - \frac{v^2}{\Lambda^2} \frac{\sum_{k} V_{ck} C_{ledq}^{(ji3k)}}{V_{cb}}~,
\end{split}\ee
where $k=1,2,3$ and for simplicity we assumed real coefficients. In general, under the $U(2)_q$ symmetry, the contribution from the $(33)$ element and from the $(23)$ one are of the same order since, for example $C_{l q}^{3 (ij 2 3)} / C_{l q}^{3 (ij 3 3)} \sim O(V_{cb})$. In realistic fits to these observables it is thus important to keep track of all contributions, which can lead to important phenomenological consequences (see e.g.\ Ref.\,\cite{Buttazzo:2017ixm}). Nevertheless, in order to extract the indicative numerical constraints, in the following we assume \emph{down-alignment} and keep as non vanishing only the $(33)$ element of these operators.

Consistently with the rest of the note, we neglect lepton-flavour-violating terms, while we allow possible deviations from lepton-flavour universality. 
In the case of operators with electron and muon one can derive the constraints from $B \to D^{(*)} \ell \nu$ decays \cite{Jung:2018lfu}, which experimentally agree with the SM prediction.
In the case of the $\tau$ lepton, instead, the corresponding decays $B \to D^{(*)} \tau \nu$ show an interesting deviation from the SM prediction with a statistical significance of more than $4\sigma$ \cite{Lees:2013uzd,Aaij:2015yra,Huschle:2015rga,Sato:2016svk,Hirose:2016wfn}. In particular, the relevant observables are ratios between the decay modes to tau and light leptons in which hadronic uncertainties largely cancel. Combining the different measurements one can express the result as (see e.g.\ Ref.\,\cite{Buttazzo:2017ixm})
\be
	R_{D^{(*)}} \equiv \frac{\mathcal{B}(B \to D^{(*)} \tau \nu)}{\mathcal{B}(B \to D^{(*)} \ell \nu)}  \left( \frac{\mathcal{B}(B \to D^{(*)} \tau \nu)_{\rm SM}}{\mathcal{B}(B \to D^{(*)} \ell \nu)_{\rm SM}} \right)^{-1} = |1 + c_{V_L}^{\tau\tau}|^2 = 1.237 \pm 0.053~,
\ee
where we showed explicitly only the contribution from the vector operator. This corresponds to $c_{V_L}^{\tau\tau} = 0.112 \pm 0.024$.
While also the scalar operator $c_{S_R}^{\tau\tau}$ contributes to the observables above, with a different weight in $R_D$ compared to $R_{D*}$, a stronger constraint can be derived from the $B_c$ lifetime \cite{Alonso:2016oyd}, due to the chiral enhancement of the scalar contribution. This gives an approximate bound $|c_{S_R}^{\tau\tau} | \lesssim 0.39$, which should be compared to the value $c_{S_R} \approx 1.5$ that would be required to fit $R_{D*}$.
%
In the case of down-alignment, a non-vanishing contribution to $D-\bar{D}$ mixing is generated by the CKM rotation from the $O_{qq}^{1(3333)}$ and $O_{qq}^{3(3333)}$ operators:
\be
	\mathcal{L}_{\rm eff} \supset \frac{C_{qq}^{1(3333)} + C_{qq}^{3(3333)}}{\Lambda^2} (V_{ub} V^*_{cb})^2 (\bar u_L \gamma_\mu c_L)^2 + h.c.~.
\ee
This can be used to cast a limit on the combination above \cite{Isidori:2013ez}.
The limits expressed in terms of the degrees of freedom defined in this note are reported in \autoref{tab:Blimits}.

\begin{table}
%\centering
\begin{tabular}{lllll}
\multicolumn{2}{l}{Four-heavy} & 
	\\\hline\noalign{\vskip1mm}
	$\ccc{+}{QQ}{}$
	& \warsaw{\cc{1}{qq}{3333}+\cc{3}{qq}{3333}}
	& $[-21, 21]$ \cite{Isidori:2013ez}
	& \multicolumn{2}{l}{(from $D-\bar{D}$, with down-alignment)}\\
	$ \ccc[\tilde]{+}{QQ}{}$
	& \warsaw{\cc[\tilde]{1}{qq}{3333}+\cc[\tilde]{3}{qq}{3333}}
	& $[-0.03, 0.03]$ \cite{Isidori:2013ez}
	& \multicolumn{2}{l}{(from $B_s-\bar{B}_s$, with up-alignment)}
\\[3mm]
\multicolumn{2}{l}{Two-heavy-two-lepton}
	& $\ell = e$
	& $\ell = \mu$
	& $\ell = \tau$ 
	\\\hline\noalign{\vskip1mm}
$\ccc{3}{Ql}{\ell}$
	& \warsaw{\cc{3}{lq}{\ell\ell33}}
	& $[-0.57, 0.22]$ \cite{Jung:2018lfu}
	& $[-0.22, 0.57]$ \cite{Jung:2018lfu}
	& $ -1.85 \pm 0.40$ \cite{Buttazzo:2017ixm} \\
$\ccc{S}{b}{\ell}$
	& \warsaw{\Re\{\cc{}{ledq}{\ell\ell33}\}}
	& $[-10, 10]$ \cite{Jung:2018lfu}
	& $[-17, 13]$ \cite{Jung:2018lfu}
	& $[-13,13]$ \cite{Alonso:2016oyd}\\[.5mm]\hline
\end{tabular}
\caption{Indicative limits on top-quark operators arising from semileptonic $B$ decays and heavy meson oscillations and for $\Lambda=1\,$TeV.}
\label{tab:Blimits}
\end{table}

With the choice of down-alignment, contributions to $b \to s$ transitions, such as in $B \to K^{(*)} \ell^+ \ell^-$ decays or in $B_s - \bar{B}_s$ mixing, arise proportionally to the off-diagonal $(32)$ element of the quark flavour matrix of the various operators. While, as already mentioned, the $SU(2)_q$ flavour symmetry predicts that this element should be of $\mathcal{O}(V_{cb})$, in the simplified discussion above this was put to zero by hand, thus forbidding tree-level contributions to these processes. It is worthwhile to briefly discuss what would happen in the opposite scenario of \emph{up-alignment}, where the operators are written in terms of $\tilde q_i = (u_{L, i},  V_{ij} d_{L, j})^T$ and only the (33) component of the flavour matrix is left non-vanishing. In this case the $b\to d_i$ transition would arise proportionally to $V_{ib}$, while there would be no $b \to c$ charged-current transition (which, in this case, would be strictly proportional to the (32) element of the flavour matrices). $\Delta F = 2$ processes in the down sector, therefore, put strong limits at tree level on four-quark operators:
\be
	\mathcal{L}_{\rm eff} \supset \frac{\tilde C_{qq}^{1(3333)} + \tilde C_{qq}^{3(3333)}}{\Lambda^2} \left[ (V_{ts} V^*_{tb})^2 (\bar b_L \gamma_\mu s_L)^2 + (V_{td} V^*_{tb})^2 (\bar b_L \gamma_\mu d_L)^2 + (V_{td} V^*_{ts})^2 (\bar s_L \gamma_\mu d_L)^2 \right]+ h.c.~.
\ee
In particular, the strongest constraint on the overall coefficient is from $B_s - \bar B_s$ mixing \cite{Isidori:2013ez}.
The measurement of various $b \to s \ell^+ \ell^-$ transitions would instead constrain the two-quark-two-lepton operators $\tilde O_{lq}^{1(ij33)}$, $\tilde O_{lq}^{3(ij33)}$, and $\tilde O_{eq}^{(ij33)}$ (see e.g.\ Refs.\,\cite{Altmannshofer:2015sma, Descotes-Genon:2015uva}).
Loop-level contributions to $B_s$ mixing and electroweak precision measurements can instead be used to put constraints on the $\tilde O_{\varphi q}^{1(33)}$, $\tilde O_{\varphi q}^{3(33)}$, and $\tilde O_{\varphi u}^{(33)}$ operators \cite{Brod:2014hsa}.
It is clear that both up- and down-alignment are very particular cases. Experimentally, since limits from $\Delta F = 2$ processes are stronger in the down sector, the down-alignment case might be preferred. Generic models of flavour are expected to interpolate between the two scenarios, therefore a more general analysis, which is well beyond the purpose of the present note, would be advisable.


 %--------------------------------------------------------------------------------
\section{Constraints from high-$p_T$ di-lepton searches}
 
\begin{table}
%\centering
\begin{tabular}{l@{\;}l lll}
\multicolumn{2}{l}{Two-heavy-two-lepton} & $\ell = e$ & $\ell = \mu$ & $\ell = \tau$ 
	\\\hline\noalign{\vskip1mm}
$\ccc{-}{Ql}{\ell}+2\,\ccc{3}{Ql}{\ell}$
	& \warsaw{\cc{1}{lq}{\ell\ell33}+\cc{3}{lq}{\ell\ell33}}
	& $[-0.32, 0.20]$ \cite{Greljo:2017vvb}
	& $[-0.43, 0.34]$ \cite{Greljo:2017vvb}
	& $[-2.6, 2.6]$ \cite{Faroughy:2016osc} \\
$\ccc{}{Qe}{\ell}$
	& \warsaw{\cc{}{eq}{\ell\ell33}}
	& $[-0.24, 0.28]$ \cite{Greljo:2017vvb}
	& $[-0.38, 0.40]$ \cite{Greljo:2017vvb}  & -- \\
%$C_{ld}^{(\ell\ell33)}$ & $[-0.27, 0.25]$ \cite{Greljo:2017vvb} & $[-0.40, 0.38]$ \cite{Greljo:2017vvb}  & -- \\
%$C_{ed}^{(\ell\ell33)}$ & $[-0.29, 0.23]$ \cite{Greljo:2017vvb} & $[-0.41, 0.37]$ \cite{Greljo:2017vvb}  & -- \\
$\ccc{S}{b}{\ell}$
	& \warsaw{\Re\{\cc{}{ledq}{\ell\ell33}\}}
	& --
	& --
	& $[-1.9, 1.9]$ \cite{Faroughy:2016osc} \\[.5mm]\hline
\end{tabular}
\caption{Indicative limits on top-quark operators arising from dilepton pair production at the LHC and with $\Lambda=1\,$TeV.}
\label{tab:dileptonlimits}
\end{table}

It is well known that the high-energy tail of $2 \to 2$ scattering processes is very sensitive to effective operators, since it allows to use the growth with energy of the new physics amplitudes to increase the signal over background ratio.
In particular, let us focus here on fermion-fermion scattering processes at the LHC such as $q \bar q \to q \bar q$ and $q \bar q \to \ell^+ \ell^-$. These have been shown to provide very strong constraints on four-fermion operators \cite{Cirigliano:2012ab, Faroughy:2016osc, Farina:2016rws, Greljo:2017vvb, Alioli:2017jdo, Alioli:2017nzr}. Since the sensitivity comes from the high energy tail, the issue of the validity of the EFT expansion is a crucial one to be addressed in this case. In particular, the underlying assumption for these bounds to be valid is that the maximal centre of mass energy from which the sensitivity is gained, $E_{\rm max}$, has to be much smaller than the mass scale of the heavy states which have been integrated out, $E_{\rm max} \ll M_{\rm NP}$. This issue has been studied in detail in the references above and it has been shown that there exist models in which the new states are heavy enough for the EFT approach to be valid, and for which the constraints obtained are relevant. For example, in Ref.\,\cite{Greljo:2017vvb}, it has been shown that $q \bar q \to e^+ e^-, \mu^+ \mu^-$ processes can be used to put significant constraints on some models addressing the neutral-current $B$-physics anomalies. Instead, the EFT interpretation of the limits in the $\tau\tau$ final state \cite{Faroughy:2016osc} should be taken as indicative only, since the mass scale of new physics cannot be too high in that case. Nevertheless, comparing with the limits on explicit models \cite{Faroughy:2016osc} shows that the EFT ones still provide a good first-order indication.
In \autoref{tab:dileptonlimits} we report the limits from \cite{Faroughy:2016osc,Greljo:2017vvb} on the two-quark-two-lepton operators involving third generation quarks, in particular the $b$ quark since it is the only one accessible in the initial state.
 
 %--------------------------------------------------------------------------------
%\section{RGE effects to EWPT and $\tau$ LFU data}

%By RG evolution, some of the the operators studied here mix into the operators affecting $Z$ and $W$ couplings to fermions, some of which have been very precisely measured at LEP-1 and in other low-energy measurements such as $\tau$ decays \cite{Feruglio:2016gvd, Feruglio:2017rjo}.
%Since the numerically largest RGE effect is the one proportional to the top-quark Yukawa coupling, in the following we consider only these contributions, neglecting those proportional to the SM electroweak couplings.
%Using the RGE equations from \cite{Jenkins:2013wua} and the results from \cite{Feruglio:2017rjo} one gets
%\bea
%	\Delta g_{\nu_\ell} &\supset& \frac{N_c y_t^2}{16 \pi^2} \frac{v^2}{\Lambda^2} \log \frac{\Lambda}{m_t} \left( C_{l q}^{3 (\ell\ell 3 3)} + C_{\ell q}^{1 (\ell\ell 3 3)} - C_{\ell u}^{(\ell\ell 3 3)} \right) ~, \nonumber\\
%	\Delta g_{\ell_L} &\supset& -\frac{N_c y_t^2}{16 \pi^2} \frac{v^2}{\Lambda^2} \log \frac{\Lambda}{m_t} \left( C_{l q}^{3 (\ell\ell 3 3)} - C_{\ell q}^{1 (\ell\ell 3 3)} \right)  ~, \nonumber\\
%	\Delta g_{\ell_R} &\supset& -\frac{N_c y_t^2}{16 \pi^2} \frac{v^2}{\Lambda^2} \log \frac{\Lambda}{m_t} \left( C_{e u}^{(\ell\ell 3 3)} - C_{eq}^{(\ell\ell 3 3)} \right) ~, \\
%	\Delta g_{\ell}^W &\supset& \frac{N_c y_t^2}{16 \pi^2} \frac{v^2}{\Lambda^2} \log \frac{\Lambda}{m_t} \left( C_{\ell q}^{3 (\ell\ell 3 3)} \right) ~. \nonumber
%\eea
%The experimental limits can be taken from \cite{ALEPH:2005ab} and constrain the expressions above at the permil level.


%%%%%%%%%%%%%%%%%%%%%%%%%%%%%%%%%
%%%%%%%%%%%%%%%%%%%%%%%%%%%%%%%%%
%\bibliographystyle{JHEP}

%{\small
%\bibliography{biblio}}

%\end{document}


}

 
