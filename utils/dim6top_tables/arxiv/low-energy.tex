% mainfile: eft_note.tex

%
% from Wouter Dekens, Jordy de Vries, Vicenzo Cirigliano, Emanuele Mereghetti
%
\subsection{Indirect constraints from low-energy probes of CP violation}
In addition to limits from direct observables, complementary constraints can be derived from low-energy measurements.
Such indirect observables do not involve resonant top quarks, 
but they are affected, in some cases very significantly, by virtual top quarks and as such can be used to probe SMEFT top-quark operators.  
Here we focus on a subset of SMEFT operators, namely the operators defined in \autoref{app:dof} that violate CP, which are mainly constrained by electric-dipole-moment (EDM) experiments and asymmetry measurements in $B\to X_s\gamma$.  
%We first adhere to the baseline flavour assumptions of \autoref{sec:flavour}, and discuss the contributions that do not require insertions of light Yukawa couplings or off-diagonal CKM elements. After discussing the limits in this case, we briefly comment on the contributions that do rely on these small couplings. 
Although we do not change the flavour structure of the operators themselves, we will deviate slightly from the baseline flavour assumptions of \autoref{sec:flavour}. In particular, we restore the off-diagonal CKM matrix elements and the light Yukawa couplings. Although these small SM parameters can be neglected in direct probes, they give rise to loop diagrams that, in some cases, induce the dominant contributions to low-energy probes. Indirect constraints on the CP-even parts, coming from electroweak precision data, are discussed in the next subsection.



As not all readers might be familiar with EDM phenomenology we give a brief introduction here. EDMs of leptons, nucleons, atoms, and molecules are probes of flavour-diagonal CPV that suffer from essentially no SM background. While the CKM mechanism predicts nonzero EDMs, they are orders of magnitude below the current experimental limits.  At present, the strongest EDM constraints arise from measurements on three different systems: the neutron, the ${}^{199}$Hg atom, and the polar molecule ThO. The limit on the latter can be interpreted (with care) as a limit on the electron EDM. In order to interpret measurements on these complicated systems in terms of the SMEFT Wilson coefficients, several steps need to be taken. First of all, the SMEFT operators must be evolved to lower energies, typically up to a scale of a few GeV where QCD is still perturbative. At that point, the SMEFT operators are matched to an effective Lagrangian describing the dynamics of the relevant low-energy degrees of freedom such as nucleons, pions, photons, and electrons. This effective Lagrangian is then used to calculate the EDMs of nucleons, nuclei, atoms, and molecules. A detailed discussion is beyond the scope of this note, and below we briefly describe how to connect the limit on the neutron and electron EDM to the SMEFT Wilson coefficients.
\newline

We start by discussing the limits from the neutron EDM. This observable obtains contributions from several CPV operators, namely four of the top-Higgs couplings, $c_{t\varphi}^{}$, $c_{tG}^{}$, $c_{bW}^{}$, and $c_{\varphi tb}^{}$, as well as the four-quark operator coefficients, $c_{QtQb}^{1,8}$.
To estimate the limits, we
%a subset of the CPV operators, namely four of the top-Higgs couplings, $c_{t\varphi}^{}$, $c_{tG}^{}$, $c_{bW}^{}$, and $c_{\varphi tb}^{}$, as well as the four-fermion operator coefficients, $c_{QtQb}^{1,8}$, $c_{t,b}^{S(e)}$, and $c_{t}^{T(e)}$.
evolve these operators, by using renormalization-group equations (RGEs), from the scale of new physics, $\Lambda$, to the scale of the top-quark mass. At this scale we integrate out the top quark. Here both $c_{t\varphi}^{}$ and $c_{tG}^{}$ induce a threshold contribution to a purely gluonic CPV operator without top quarks, the so-called Weinberg operator 
\begin{equation}
\mathcal L_W =  \frac{g_S}{6}\frac{d_W}{\Lambda^2}f^{abc} \epsilon^{\mu\nu\alpha\beta}G^a_{\alpha\beta}G^b_{\mu\rho}G^{c\,\rho}_{\nu}\,,
\end{equation}
with \cite{Dicus:1989va,Weinberg:1989dx,BraatenPRL,Boyd:1990bx,Kamenik:2011dk, Brod:2013cka}
\begin{equation}
d_W(m_t) = \frac{g_S^2}{64\pi^4}\frac{v}{\sqrt{2} m_t}h(m_t, m_h)\,c_{t\varphi}^{I}
+ \frac{g_S^2}{(4 \pi)^2}\frac{v}{\sqrt{2} m_t }\, c_{tG}^{I}\,,
\label{eq:matching}\end{equation}
where $h(m_t,m_h)\simeq 0.05$ is a finite two-loop integral. Note that $c^I_{t\varphi}$ also contributes to the EDMs of light quarks, proportional to their Yukawa couplings, through two-loop Barr-Zee diagrams \cite{Barr:1990vd,Gunion:1990iv,Abe:2013qla,Jung:2013hka}. We include this effect in the limits discussed below. 


In addition, $c_{\varphi tb}^{}$, $c_{bW}^{}$, and $c_{QtQb}^{1,8}$ contribute to the Weinberg operator by first inducing  the bottom chromo-EDM, $O_{dG}^{(33)}$. For the four-quark operators and $O_{dW}^{(33)}$ the bottom chromo-EDM is induced through renormalization-group evolution \cite{Dekens:2013zca, Alonso:2013hga, Cirigliano:2016nyn} between $\mu=\Lambda$ and $\mu=m_t$, while  $O_{\varphi tb}^{}$ only provides a matching contribution
\begin{eqnarray}
\frac{d {\rm Im}\,C_{dG}^{(33)}}{d\ln\mu}&=&\frac{\sqrt{2}}{(4\pi)^2}\frac{m_t}{v}\left(c_{QtQb}^{1I}
-\frac{1}{2N_C}c_{QtQb}^{8I}\right)
+\frac{2}{(4\pi)^2}\left[3g_W^2+\frac{1}{3}g_Y^2\right] c_{bW}^{I}\dots\,,\nonumber\\
{\rm Im}\,C_{dG}^{(33)}(m_t^-)&=&{\rm Im}\,C_{dG}^{(33)}(m_t^+)+\frac{1}{\sqrt{2}(4\pi)^2}\frac{m_t}{v}f_W\,  c_{\varphi tb}^{I}
\,,
\end{eqnarray}
where $f_W\simeq 0.7$ is  a loop function \cite{Alioli:2017ces} and the ellipsis stands for the self-renormalization.  At the bottom mass scale $O_{dG}^{(33)}$ then induces a contribution to the Weinberg operator 
\begin{equation}
d_W(m_b^-) =d_W(m_b^+) + \frac{g_S^2}{(4\pi)^2}\frac{v}{\sqrt{2} m_b }\, \mathrm{Im}(C^{(33)}_{dG}(m_b^+))\,.
\end{equation}
The Weinberg operator can now be evolved to lower energies and, around the QCD scale, be matched to hadronic operators. In particular, it induces a contribution to the neutron EDM
\begin{equation}
|d_n| \simeq (50\,\mathrm{MeV})\,e g_s(1\,\mathrm{GeV})\,d_W(1\,\mathrm{GeV})/\Lambda^2\,.\label{neutronEDM}
\end{equation}
The required hadronic matrix element  suffers from large uncertainties and here we have taken the average of various estimates \cite{Pospelov_Weinberg, Weinberg:1989dx, deVries:2010ah}. The impact of hadronic and nuclear uncertainties on low-energy precision constraints on the SMEFT operators can be significant and has been discussed in detail in Refs.~\cite{Chien:2015xha, Cirigliano:2016nyn}. 
The constraints that result from employing \autoref{neutronEDM} and the experimental limit,  $d_n < 3.0 \cdot 10^{-13}$ e fm \cite{Afach:2015sja,Baker:2006ts}, are collected in \autoref{Tab:EDMlimits}.
\newline

Moving on to the electron EDM, there are again several operators that contribute, namely, three  top-Higgs couplings $c^{I}_{tA, tW,t\varphi}$, as well as the semi-leptonic operators $c_{t,b}^{S(e)}$ and  $c_{t}^{T(e)}$.
Of the semi-leptonic operators the tensor operator induces the electron EDM through a single top loop, while the scalar ones require additional loops. The relevant RGEs are given by \cite{Cirigliano:2016nyn, Alonso:2013hga},
\begin{equation}
\frac{d }{d\ln\mu}\begin{pmatrix} d_e\\
c_{t}^{SI(e)}/\Lambda^2\\c_{t}^{TI(e)}/\Lambda^2\\c_{b}^{SI(e)}/\Lambda^2\end{pmatrix}
=\frac{1}{(4\pi)^2}\begin{pmatrix}
0 &0& -16N_C Q_t m_t&0\\
0 & -6C_Fg_S^2&0&2(1-N_c)y_ty_b\\
0 & \frac{3}{8}g_W^2+\frac{5}{8} g_Y^2 &2C_F g_S^2&0\\
0&0&0&-6C_F g_S^2\\
\end{pmatrix}\cdot \begin{pmatrix} d_e\\
c_{t}^{SI(e)}/\Lambda^2\\c_{t}^{TI(e)}/\Lambda^2\\c_{b}^{SI(e)}/\Lambda^2\end{pmatrix}\,,
\end{equation}
where $C_F=(N_C^2-1)/2N_C$, $y_{b,t}=m_{b,t}\sqrt{2}/v$, $Q_f$ stands for the electric charge, and we only kept the electroweak terms that are relevant for the mixing of $c_{t,b}^{SI(e)}$ into $c_{t}^{TI(e)}$. The solution of the above equations provides the leading logarithmic contributions to the electron EDM. These, combined with $d_e\leq 8.7\cdot 10^{-16}\,e$ fm \cite{Baron:2013eja}, give stringent constraints, which are again collected in \autoref{Tab:EDMlimits}.
The same loops that induce the electron EDM also give a contribution to the electron anomalous magnetic moment, proportional to the real part of the semileptonic couplings. 
Although the resulting limits are weaker than the EDM limits they are still significant for two of the couplings, we obtain $|c_{t}^{S(e)}|\lesssim 2\cdot 10^{-2}$ and  $|c_{t}^{T(e)}|\lesssim 3\cdot 10^{-5}$.


Of the top-Higgs couplings, $c_{t\varphi}^{I}$ generates the electron EDM through two-loop Barr-Zee diagrams \cite{Barr:1990vd,Gunion:1990iv,Abe:2013qla,Jung:2013hka}, giving a stronger limit than the neutron EDM. 
In addition, when we evolve the $c^{I}_{tA, tW}$ couplings from the scale of new physics, $\Lambda$, to lower energies they first mix into CPV Higgs-gauge couplings of the form $(\varphi^\dagger \varphi)\tilde X_{\mu\nu} X^{\mu\nu}$ \cite{Cirigliano:2016nyn,Cirigliano:2016njn,Fuyuto:2017xup}, where $X$ denotes an $SU(2)$ or $U(1)$ gauge-field strength. In a second step these gauge-Higgs couplings mix into the electron and light-quark EDMs. This last step is proportional to the Yukawa couplings of the light fields leading to a strong suppression. Nevertheless, the experimental limit on the electron EDM is sufficiently strong to overwhelm the other probes of the CPV components of these two top-quark dipoles \cite{Cirigliano:2016nyn,Cirigliano:2016njn}.  
\newline



\begin{table}
\renewcommand{\arraystretch}{1.2}%
\begin{tabular}{@{}l@{\:}llllll@{}}
\multicolumn{2}{@{}l}{Four-heavy} & \multicolumn{4}{l}{}\\
\hline\noalign{\vskip 1mm}
$c_{QtQb}^{1I}$
	& \warsaw{\Im\{\cc{1}{quqd}{3333}\}} 
	& $[-3.4,\, 3.4]\cdot 10^{-3}$ &($d_n$)&&\\
$c_{QtQb}^{8I}$
	& \warsaw{\Im\{\cc{8}{quqd}{3333}\}} 
	&$[-2.2,\, 2.2]\cdot 10^{-2}$&($d_n$) &&\\[2mm]
\multicolumn{2}{@{}l}{Two-heavy} & \\\hline\noalign{\vskip 1mm}
$c_{t\varphi}^{I}$
	& \warsaw{\Im\{\cc{}{u\varphi}{33}\}}
	&$[-3.7,\,3.7]$&($d_n$) &\quad$[-0.18,\, 0.18]$&($d_e$)\\
$c_{\varphi tb}^{I}$
	& \warsaw{\Im\{\cc{}{\varphi ud}{33}\}}
	&$[-0.019,\, 0.019]$&($d_n$)&\quad $[-0.052,\, 0.052]$&($B\to X_s\gamma$)\\
$c_{ tW}^{I}$
	& \warsaw{\Im\{\cc{}{uW}{33}\}}
	&$[-8.1,\, 8.1]\cdot 10^{-3}$&($d_e$)&\quad$[-2.4,\, 4.5]$&($B\to X_s\gamma$)\\
$c_{tA}^{I}$
	& \warsaw{\Im\{\cw\cc{}{uB}{33}+\sw\cc{}{uW}{33}\}}
	&$[-6.3,\, 6.3]\cdot 10^{-3}$&($d_e$)&\quad$[-9.0,\, 5.0]$&($B\to X_s\gamma$)\\
$c_{bW}^{I}$
	& \warsaw{\Im\{\cc{}{dW}{33}\}}
	& $[-5.5,\, 5.5]\cdot 10^{-4}$&($d_n$)&\quad $[-4.3,\, 2.3]\cdot 10^{-2}$&($B\to X_s\gamma$)\\
$c_{tG}^{I}$
	& \warsaw{\Im\{\cc{}{uG}{33}\}}
	& $[-6.9,\,6.9]\cdot 10^{-3}$&($d_n$)\\[2mm]
\multicolumn{2}{@{}l}{Two-heavy-two-lepton} & \\\hline\noalign{\vskip 1mm}
$c_{t}^{SI(e)}$
	& \warsaw{\Im\{\cc{1}{lequ}{1133}\}}
	& $[-5.5,\, 5.5]\cdot 10^{-8}$&($d_e$)\\
$c_{t}^{TI(e)}$
	& \warsaw{\Im\{\cc{3}{lequ}{1133}\}}
	& $[-8.0,\, 8.0]\cdot 10^{-11}$&($d_e$)\\
$c_{b}^{SI(e)}$
	& \warsaw{\Im\{\cc{}{ledq}{1133}\}}
	& $[-2.5,\, 2.5]\cdot 10^{-4}$&($d_e$)\\\hline
\end{tabular}
\caption{Constraints from the electron and neutron EDMs as well as $A_{CP}(B\to X_s\gamma)$. Here we turn on one coupling at a time and assume  $\Lambda =1$ TeV. The source of the constraints are indicated in brackets.}\label{Tab:EDMlimits}
\end{table}

Finally, we briefly discuss limits from rare $B$ decays. 
%Additional constraints can be set once one takes into account the nonzero masses of the first two generations and off-diagonal CKM elements.
At the one-loop level, the couplings $c^{I}_{\varphi tb, bW, tA, tW}$ give contributions to flavour-changing dipole operators that mediate $b\rightarrow s $ transitions proportional to the CKM element $V_{ts}\simeq 0.04$. 
The contributions of these flavour-changing operators to the CP asymmetry $A_{CP}(B\to X_s\gamma)$ \cite{Benzke:2010tq}, together with the experimental measurement \cite{Amhis:2016xyh}, can  be used set the limits in \autoref{Tab:EDMlimits}.
It should be noted that measurements of the branching ratio can constrain the real parts of these couplings as well. This leads to limits which are typically a factor of a few stronger than those on the imaginary parts, see, for example,  Refs.~\cite{Grzadkowski:2008mf,Drobnak:2011aa,Drobnak:2011wj,Kamenik:2011dk}.

%Furthermore, once one includes light-fermion masses, the operators that previously only induced the Weinberg operator can also contribute to the EDMs of light fermions, proportional to their Yukawa couplings. For most of the operators this does not dramatically change the constraints. The only exception is $c_{t\varphi}^{I}$, which now generates the electron EDM through two-loop Barr-Zee diagrams \cite{Barr:1990vd,Gunion:1990iv,Abe:2013qla,Jung:2013hka}, giving a stronger limit than the neutron EDM. Finally, the $c^{I}_{tA, tW}$ couplings, which are only weakly constrained by $B\to X_s\gamma$, can now be constrained by the electron EDM. Evolving  these two top-quark dipole operators from the scale of new physics, $\Lambda$, to lower energies they first mix into CPV Higgs-gauge couplings of the form $(\varphi^\dagger \varphi)\tilde X_{\mu\nu} X^{\mu\nu}$ \cite{Cirigliano:2016nyn,Cirigliano:2016njn,Fuyuto:2017xup}, where $X$ denotes an $SU(2)$ or $U(1)$ gauge-field strength. In the next step the gauge-Higgs couplings mix into the electron and light-quark EDMs. This last step is proportional to the Yukawa couplings of the light fields leading to a strong suppression. Nevertheless, the experimental limit on the electron EDM is sufficiently strong to overwhelm the other probes of the CPV components of these two top-quark dipoles \cite{Cirigliano:2016nyn,Cirigliano:2016njn}.  

\subsubsection*{Summary}
All the above discussed constraints are collected in \autoref{Tab:EDMlimits}. With the exception of $c^{I}_{t\varphi}$, the CPV coefficients are constrained at the percent level or stronger by EDM experiments. The semi-leptonic operators are more stringently constrained, which is mainly due to the fact that their contribution to $d_e$ is proportional to $m_t$ (where one would naively expect $m_e$). Instead, the constraints from $B\to X_s\gamma$ are particularly strong for the $c_{bW}^{I}$ and $c_{\varphi tb}^{I}$ couplings because of an $m_t/m_b$ enhancement.
In most cases, the constraints on the imaginary parts are stronger than the corresponding limits on the real parts of the top-quark couplings. As a result, it will be difficult to reach a similar sensitivity by studying CPV observables at the LHC. 

The interpretation of these low-energy constraints requires some care however. In deriving these limits we have assumed one dimension-six operator to be present at the scale $\Lambda$ at a time. This assumption is no longer valid if multiple top operators are important at the scale $\Lambda$, or if one makes less restrictive assumptions about the flavor structure, such as a non-linear flavor symmetry \cite{Kagan:2009bn}. A global analysis involving all top-quark operators, for example, would leave some combinations of operator coefficients unconstrained~\cite{Cirigliano:2016nyn}. The difference between \emph{individual} and \emph{global} constraints is typically large for the CPV components as only a handful of sensitive low-energy measurements exist, in contrast to a much larger range of high-energy measurements of the real components. In a global setting, collider constraints on the CPV Wilson coefficients are therefore necessary to bound unconstrained directions in the parameter space. An example is the recent ATLAS measurement~\cite{Aaboud:2017yqf} of a CPV phase in $t\rightarrow bW$ decays which significantly impacts the global fit of CPV top-Higgs interactions~\cite{Cirigliano:2016nyn,Alioli:2017ces}.

In addition, the low-energy observables get contributions from SMEFT operators that do not involve top quarks. For example, if an electron EDM is generated at the scale $\Lambda$ with exactly the right size, it could weaken the limits on the semi-leptonic operators significantly. 
In fact, the leading logarithmic contributions we considered here result from divergent loops and require non-top operators, such as $d_e$, to absorb the divergences. In these cases one might expect $d_e(\Lambda)\neq 0$, which could in principle lead to cancellations and weakened limits. These cancellations have to be very severe in order to avoid the strong low-energy constraints. In any case, in order to evade the low-energy limits strong correlations between SMEFT operators are required, and this would  pose highly non-trivial constraints on models of beyond-the-SM physics. 
